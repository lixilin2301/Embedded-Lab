\section{Future Work}
Several improvements could be done to our implementation.

\subsection{Dynamic configuration}

At the moment the default calculation configuration is \emph{DSP + NEON}.
This 'static' configuration benefits only when the input size is greater than a specific value as shown in the analysis section.
However, better overall performance can be achieved when the 
configuration is to be dynamically chosen according to the input data size.
Looking at Figure~\ref{fig:speedup_plot} on page~\pageref{fig:speedup_plot} again, when the received input size is less than 90, the calculation can be performed by NEON only,
whereas when the input size is greater than 90, then the calculation can be handled by the DSP + NEON configuration. This improvement can be simply done without major changes to the coding infrastructure by adding an \textit{if statement} to choose between DSP+NEON and NEON only configuration at run-time.

\subsection{Reducing communication overhead}

The current implementation sends a fixed number of messages over the communication links. That time can be extremely wasteful, e.g. if the input matrix size is 64, we could limit ourselves to one packet being sent to the DSP, 
however at the moment we are sending three additional packets that contain no useful information. Hence the performance can be improved by reducing the communication time if we dynamically decide how many packets are sufficient enough and send no more than that.
Alternatively, 
another area of exploration that could be experimented is to send more messages with smaller size to minimize communication time. The idea is that by having smaller packets we are able to more efficiently 'packetize' the data and send it over the link without fragmentation.