\section{Future Work}
In our experiments, the calculation is always performed by both DSP and NEON.
This kind of static allocation benefits only when the input size is greater than a specific value.
Obtained from the analysis, better performance can be achieved when the 
calculation is to be allocated on different cores dynamically according to the input data size.

Based on Figure 6, when the received input size is less than 90, the calculation is performed by NEON only,
while when the input size is greater than 90, then the calculation is handled by both DSP and NEON.
Moreover, the performance can be further improved if we could configure the number of messages sending to DSP.
Since we will not reach the maximum capacity of sending message size until certain input data size, 
less messages are able to send to DSP in order to reduce the communication costs. Alternatively, 
during our experiments, it is also possible to send more messages with smaller size to minimize communication time.

In conclusion, more improvements are to be achieved when dynamic allocation is to be implemented on core selection and 
DSP message generation.