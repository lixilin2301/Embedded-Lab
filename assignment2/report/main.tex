\documentclass[10pt,final,journal]{IEEEtran}
\usepackage[english]{babel}
\usepackage{listings}
\usepackage{graphicx}
\usepackage{caption}
\usepackage{subcaption}
\usepackage{amsmath}
\usepackage{color}
\usepackage{cite}
\usepackage{url}
\usepackage{algorithm}
\usepackage{algpseudocode}
\usepackage{hyperref}

\graphicspath{{./images/}}

\usepackage[margin=1in]{geometry}

\definecolor{mygray}{rgb}{0.4,0.4,0.4}
\definecolor{mygreen}{rgb}{0,0.8,0.6}
\definecolor{myorange}{rgb}{1.0,0.4,0}

\definecolor{lightgray}{rgb}{.9,.9,.9}
\definecolor{darkgray}{rgb}{.4,.4,.4}
\definecolor{purple}{rgb}{0.65, 0.12, 0.82}

\DeclareMathSizes{10}{10}{10}{10}

\lstset{
backgroundcolor=\color{lightgray},
extendedchars=true,
basicstyle=\footnotesize\ttfamily,
showstringspaces=false,
showspaces=false,
numbers=left,
numberstyle=\footnotesize,
numbersep=9pt,
tabsize=2,
breaklines=true,
showtabs=false,
captionpos=b
keywordstyle=\color{blue}\bfseries,
ndkeywordstyle=\color{darkgray}\bfseries,
identifierstyle=\color{black},
commentstyle=\color{purple}\ttfamily,
stringstyle=\color{red}\ttfamily,
}

\title{Embedded Lab Assignment 2\\\small{Group 7}}
\author{
		\IEEEauthorblockN{
			Haji~Akhundov\IEEEauthorrefmark{1}
			Misael~Hernandez~Leal\IEEEauthorrefmark{2}
			Fei~Tan\IEEEauthorrefmark{3}
			Koray~Yanik\IEEEauthorrefmark{4}
			Muneeb~Yousaf\IEEEauthorrefmark{5}
		}

		\IEEEauthorblockA{
			\IEEEauthorrefmark{1}h.akhundov@student.tudelft.nl			\small{4390547} \and
			\IEEEauthorrefmark{2}m.a.hernandezleal@student.tudelft.nl 	\small{4423615} \and 	\\
			\IEEEauthorrefmark{3}f.tan@student.tudelft.nl 				\small{4405722} \and
			\IEEEauthorrefmark{4}k.i.m.yanik@student.tudelft.nl 		\small{4382781} \and 	\\
			\IEEEauthorrefmark{5}m.m.yousaf@student.tudelft.nl 			\small{4411129}
		}
}

\date{\today}

\begin{document}

\nocite{*}

\maketitle

\begin{abstract}
This report describes the optimization process of an existing sequential image processing algorithm, namely edge detection on a heterogeneous system with multiple processing units.

Given the reference application, the code is modified such that it runs on an embedded platform, utilizing a digital signal processor  and parallel computation extensions to achieve a speedup over a purely sequential reference implementation.
Specifically the application was implemented on a TI processor that includes an ARM core with NEON support and a DSP.

Profiling the reference application shows that 80\% of the processing time is spent in the gaussian filtering function. Further investigation revealed that the work load of this function could be distributed over the available processing units.
By carefully load balancing this function among the available processing units, a speedup of a factor of two and above was achieved.

\end{abstract}

\newpage
\section{Introduction}

The purpose of this assignment is to perform matrix multiplication on a multi-core heterogeneous computing platform. The given platform is the Beagle Board\footnote{http://beagleboard.org/} that has a general purpose ARM processor (from now on referred to as the \emph{GPP}) and a digital signal processor (\emph{DSP}) that would be used in this lab for matrix multiplication and eventually obtaining a speed up by performing computations in parallel. Luckily matrix multiplication has a large number of multiplication and addition operations that can be parallelized. In this report we will go through describing each component and the communication model that we have come up with to communicate our matrix structures in the heterogeneous environment. We will see that the communication has overhead and imposes additional delays. Furthermore every core has a hardware limitation in terms of how many operations can be performed at the same time, which defines the upper bound on how much speed-up can be obtained. At last results and their discussion is shown as well as a section on future improvements.


\section{Profiling}
\label{sec:profiling}
The memory and time complexity of any application is measured by the profiling. In this assignment, the \emph{MCProf} profiling tool is being used for profiling of the \emph{Canny Edge Detection} application. MCProf is a runtime Memory and Communication profiler which generates detailed application profiles in terms of memory access patterns and data-communication at function and loop-level granularity.

\begin{figure}[h]
\centering
\includegraphics[width=0.4\textwidth]{tiger_profile}
\caption{Application profiling for \textit{tiger.png}. \textit{Some output omitted for brevity}}
\label{fig:prof}
\end{figure}

Profiling of this application was performed for the given set of sample pictures and the results showed us that the \emph{gaussian-smooth} function consumes almost 80\% of the processing time for each input picture. Moreover, most of the memory accesses are also done by this function. Therefore, this function has become the center of our attention and was subject to acceleration. The profiling result with the \emph{tiger.pgm} as an input image is shown in figure~\ref{fig:prof}. Note that the input image has very little impact on the results' deviations.
\textit{The profiling results of the rest of the pictures are attached in the appendix.}


\section{Implementation}
Obtaining the results from profiling we see that approximately 80\% of the time is spent on executing the \textit{gaussian\_smooth} function. Observing the code of this function we see that it is divided into two for loops (also seen in profiling).
First the blurring is done by convolution in the X-direction and then in the Y-direction by applying the gaussian kernel (one dimensional convolution matrix). This process is briefly shown in figure~\ref{fig:convolution}. There is a data dependency between the two convolution processes where one must be followed by the other, so simply performing one convolution in one direction on one processing unit and the other direction on a different processor will not work.

\begin{figure}
\centering
\includegraphics[width=0.4\textwidth]{drawings/gaussian_general}
\caption{Convolution process}
\label{fig:convolution}
\end{figure}

The most naive way of splitting the work between two processors and obtaining a speed-up is to compute the gaussian filtering on both processors for different parts of the image. The most simplistic way with a 50/50 load balancing is depicted in figure~\ref{fig:balancing} where the NEON is performing the convolutions on the top part of the image and the DSP performs the convolutions on the bottom part of the image. As shown in the workflow diagram in figure~\ref{fig:workflow} we then proceeded with the development of the gaussian filtering independently on each processor and later use clever techniques to split the computation between the two processors.

On the gpp side we define FRAC which is an integer value from zero to hundred that specifies the fraction of calculations that must be done on the NEON. After integration, both calculations were measured in time and a FRAC that resulted in an even balance between gpp and dsp was chosen. The two calculations have to be completed before the program can resume.

By load balancing we make sure that the synchronization happens as soon as possible and very little time is wasted on waiting for one of the processes to finish. Figure~\ref{fig:dataflow} shows the resulting dataflow of the current implementation. The subsequent sections go into more details of this process.

\begin{figure}
\centering
\includegraphics[width=0.4\textwidth]{drawings/gaussian_balancing}
\caption{Splitting the work}
\label{fig:balancing}
\end{figure}

\begin{figure}
\centering
\includegraphics[width=0.4\textwidth]{drawings/model}
\caption{Data flow}
\label{fig:dataflow}
\end{figure}

\section{DSP}
Discuss communcation with the DSP processor here.

\section{NEON}
To improve the performance of media and signal processing, 
\emph{NEON} SIMD technology is implemented in \emph{Cortex-A8} core of \emph{OMAP 3530} GPP. 
\emph{NEON} SIMD technology, which is also known as Advanced SIMD extension, 
takes the advantage of parallel operation to achieve the speed up.
\subsection{SIMD}
To understand NEON technology, the idea of SIMD is introduced at first. 
SIMD (Single Instruction Multiple Data) describes a way to perform the same operation on multiple data with same type and size in a single instruction. 
The idea of parallel operation comes from the fact that most of multimedia data are 16-bit or 8-bit wide, while the general purpose registers are 32-bit wide. 
To effectively utilize the space of registers, simultaneous computation is developed.  
\subsection{NEON Technology} 
\emph{NEON} technology, as the Advanced SIMD extension in \emph{Cortex-A8}, performs SIMD operations in group. 
NEON instructions operate on vectors stored in 64-bit or 128-bit registers, 
then vectors of elements with same type can perform the same operation on multiple items at the same time.
Figure~\ref{fig:neon} shows how multiple items are computed simultaneously. 

\begin{figure}[h]
\centering
\includegraphics[width=0.9\textwidth]{images/neon}
\caption{Parallel computing based on NEON}
\label{fig:neon}
\end{figure}

\subsection{Hardware Features}
NEON architecture has the following features\cite{hardware}:
\begin{enumerate}
\item \emph{16-Entry instruction queue}
\item \emph{32 x 64-bit general purpose registers in register file}
These registers can alternatively be viewed as 16 x 128-bit registers
\item \emph{6-stage execution pipeline}
NEON supports either integer of single precision floating point execute pipeline.
\item \emph{Load/store and permute pipeline}
\item \emph{12–Entry load data queue}
\end{enumerate}

\subsection{Implementation}
To enable the build-in intrinsics of NEON, 
\emph{$-$mfpu$=$neon}\cite{ARMoptions} is used during compiling time.
Also, header file \emph{arm\_neon.h} is included 
to support NEON intrinsics in the c file.
In our case, the incoming message contains the matrix with data of 16-bit wide, 
and then after calculation, when the final outcome is sending back to GPP, data size is 32-bit to avoid overflow.
In the following, NEON intrinsics that are used for parallel computing in our experiment are explained.
\subsubsection{Vector Data Type}
Neon defined its own data type\cite{DataType} for multiple data operation, the format is given as:
~\\ 
\textbf{ \textless type\textgreater \textless size \textgreater x\textless number of lanes\textgreater\_t}

For example, the data type we are going to use in NEON is \emph{uint32x4\_t}, 
which means the vector has four lanes, 
with each of the them containing an unsigned 32-bit integer. 
\subsubsection{NEON Intrinsics}
NEON intrisics \cite{Intrinsics}provide groups of functions for operation. 
In our case, functions related to load, multiplication and addition are used. 
\begin{enumerate}
\item \textbf{uint32x4\_t  vmovq\_n\_u32(uint32\_t value)}

This intrinsic loads all lanes of vector to the same input value. 
The input value is an unsigned 32-bit integer, 
while the four lanes being loaded each contains an unsigned 32-bit as well.


\item \textbf{int32x4\_t   vld1q\_s32(\_\_transfersize(4) int32\_t const * ptr)}

This intrinsic loads a single value from memory to all lanes.
The data stored in memory is signed 32-bit integer, 
while the four lanes each contains a signed 32-bit integer.

\item \textbf{int32x4\_t   vmlaq\_s32(int32x4\_t a, int32x4\_t b, int32x4\_t c)}

This intrinsic multiplies b by c, and accumulates the result with a in all four lanes.
The final results are then stored in four lanes as well.
\end{enumerate}

\subsubsection{Multiplication Algorithm}

In order to fully utilize the Neon resources, we have used following algorithm to compute the matrix multiplication. The following illustration shows one rows of results.
The input matrices are following:

$$
\begin{pmatrix}
 x1 	& \color{red}{x2} 	& \color{green}{x3} & \color{blue}{x4}\\
 x5 	& x6 				& x7&x8\\
 x9 	& x10 				& x11&x12\\
 x13 	& x14 				& x15&x16\\
\end{pmatrix}
\times
\begin{pmatrix}
y1&y2&y3&y4\\
y5&y6&y7&y8\\
y9&y10&y11&y12\\
y13&y14&y15&y16\\
\end{pmatrix}
$$


The output matrix would be : 
=
\begin{table}[h]
	\resizebox{\textwidth}{!}{
		$\left(
			\begin{tabular}{cccc}
				x1y1+\color{red}{x2}\color{black}y5+\color{green}{x3}\color{black}y9+\color{blue}{x4}\color{black}y13&x1y2+\color{red}{x2}\color{black}y6+ \color{green}{x3}\color{black}y10+\color{blue}{x4}\color{black}y14&x1y3+\color{red}{x2}\color{black}y7+\color{green}{x3}\color{black}y11+\color{blue}{x4}\color{black}y15&x1y4+\color{red}{x2}\color{black}y8+ \color{green}{x3}\color{black}y12+\color{blue}{x4}\color{black}y16\\
				-&-&-&-\\-&-&-&-\\-&-&-&-\\
			\end{tabular}
		\right)$
	}
\end{table}

In the output matrix ,  the  elements with the same color are multiplied by the Neon at the same time. In this way, Neon produces the results of 4 multiplications at the same time.  In the above output row, x1y1, x1y2, x1y3 and x1y4 are produces by the neon simultaneously.  But these are just the partial results for the first row of output matrix. In order to get the complete results for the first row these partial results should be added to the other partial results  of the same row.  This complete task of multiplication and addition is done by using "Multiplication and Addition" functional unit of neon. The conceptual diagram of this multiplication and addition is shown in figure~\ref{fig:neon_mult_add}.
\begin{figure}[h]
\centering 
\includegraphics[width= 0.7\textwidth]{images/MandA}
\caption{Parallel Multiplication and Addition with Neon  }
\label{fig:neon_mult_add}
\end{figure}



\input{Analysis}

\section{Conclusion}
Overall, implementing a highly parallelizable algorithm on a machine capable 
of SIMD instructions with an additional DSP is discussed. Because there is a 
lot of communication overhead in a static solution, the acquired speed-up is 
fairly low for lower input sizes. When the input size grows, this gets closer 
to the theoretical upper-bound as defined by Amdahl's Law. 


\section{Future Work}
Several improvements could be done to our implementation.

\subsection{Dynamic configuration}

At the moment the default calculation configuration is \emph{DSP + NEON}.
This 'static' configuration benefits only when the input size is greater than a specific value as shown in the analysis section.
However, better overall performance can be achieved when the 
configuration is to be dynamically chosen according to the input data size.
Looking at Figure~\ref{fig:speedup_plot} on page~\pageref{fig:speedup_plot} again, when the received input size is less than 90, the calculation can be performed by NEON only,
whereas when the input size is greater than 90, then the calculation can be handled by the DSP + NEON configuration. This improvement can be simply done without major changes to the coding infrastructure by adding an \textit{if statement} to choose between DSP+NEON and NEON only configuration at run-time.

\subsection{Reducing communication overhead}

The current implementation sends a fixed number of messages over the communication links. That time can be extremely wasteful, e.g. if the input matrix size is 64, we could limit ourselves to one packet being sent to the DSP, 
however at the moment we are sending three additional packets that contain no useful information. Hence the performance can be improved by reducing the communication time if we dynamically decide how many packets are sufficient enough and send no more than that.
Alternatively, 
another area of exploration that could be experimented is to send more messages with smaller size to minimize communication time. The idea is that by having smaller packets we are able to more efficiently 'packetize' the data and send it over the link without fragmentation.

%\bibliography{references}{}
%\bibliographystyle{unsrt}

\end{document}